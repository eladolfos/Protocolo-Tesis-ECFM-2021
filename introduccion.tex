%%% Haga el diseño que más le guste
\chapter*{INTRODUCCIÓN}

Los objetos compactos son objetos exóticos, en ellos las teorías físicas se ponen a prueba, son los mejores laboratorios naturales en los que la materia se somete a condiciones extremas. Dentro los objetos compactos tenemos a las estrellas de neutrones, enanas blancas y agujeros negros. \\

Muchos de estos objetos exóticos podemos encontrarlos en sistemas binarios, uno de esos sistemas sumamente exótico son los denominados ``Black Widow'', estos sistemas están compuesto por una estrella de neutrones que gira rápidamente y emite radiación con la cual destruye o ``devora'' a su compa\~nera la cual típicamente es una estrella de secuencia principal. De este proceso típicamente destructivo es generalmente aceptado que los p\'ulsares de milisegundo aislados son pulsares que en su momento se encontraban en un sistema de este tipo.\\


Este proyecto se enfocar\'a en el análisis óptico utilizando datos obtenidos del telescopio óptico mas grande del mundo; El Gran Telescopio Canarias, en los filtros Sloan $g', r'$, e $i'$ y de ello obtener curvas de luz en cada uno de los filtros, as\'i como las gráficas de espectro de potencia. Por \'ultimo se deducirán parámetros  del sistema tales como:  la amplitud de la variación de la curva de luz en cada una de las bandas, el periodo orbital del sistema, la temperatura en la superficie del lado irradiado por el p\'ulsar (``day-side'') y el lado no irradiado (``night-side'') y la eficiencia calorífica de la estrella compa\~nera del p\'ulsar.







