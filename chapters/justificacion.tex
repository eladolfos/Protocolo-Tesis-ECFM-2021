\chapter{JUSTIFICACIÓN}

Los sistemas binarios  denominados ``Black Widow'' son sistemas extremadamente raros en la naturaleza, formados por un p\'ulsar y otra estrella a la que \'este devora. El primer descubrimiento de un sistema de esta naturaleza fue realizado por \citet{fruchter1988} y desde entonces los astrofísicos se han esforzado en caracterizar estos sistemas. En estos sistemas, la estrella compañera es disminuida por la radiación de alta energía del púlsar y el viento de partículas relativistas hasta que eventualmente puede evaporarse por completo, de esto que sea generalmente aceptado que los p\'ulsar de milisegundo solitarios fueron en su momento sistemas de este tipo pero que eliminaron a su compa\~nera perteneciente a la secuencia principal, como muestran \citet{1974Bisnovatyi} y \citet{1982New-class-MSP-Alpar}. \\



Estos sistemas son tan raros que no existen aun libros que recopilen información sobre ellos, y hasta la fecha solo una peque\~na fracción de ellos ha sido estudiada en el rango óptico como muestran \citet{Zharikov2019}. Las observaciones ópticas son importantes para seguir los procesos de evolución, evaporación y la formación de p\'ulsares de milisegundo solitarios. Es por ello que este trabajo busca aportar conocimientos al estudio óptico de estos objetos, en particular del objeto PSR J1641+8049.  


