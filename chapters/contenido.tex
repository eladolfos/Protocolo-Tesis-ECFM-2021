\chapter{METODOLOGÍA}
Para el desarrollo del Cap\'itulo 1 se utilizaran los libros de \citet{1983-shapiro}, \citet{Camenzind2007}, \citet{Carroll2007}, \citet{karttunen2007fundamental}, \citet{1986Rybicki-Radiative-Pro-Ast}, as\'i como artículos relacionados con Black Widows tales como \citet{1988Eichler-On-Black-widow},  \citet{van2011} entre otros que proporciones los conceptos necesarios para el desarrollo de este trabajo de graduación.\\

Por otro lado para el Cap\'itulo 2 se utilizar\'a la información oficial del sitio web de El Gran Telescopio Canarias\footnote{\url{http://www.gtc.iac.es/}}, as\'i como el manual de OSIRIS (Optical System for Imaging and low-Intermediate-Resolution Integrated Spectroscopy)\footnote{\url{http://www.gtc.iac.es/instruments/osiris/media/OSIRIS-USER-MANUAL_v3_1.pdf}}.\\

Para el Cap\'itulo 3 se realizaran los datos obtenidos de El Gran Telescopio Canarias, siguiendo el procedimiento para reducción de datos provisto por \citet{AvilaOAN}, para lo cual se utilizar\'a IRAF y PYRAF as\'i como distintos scripts para realizar los ajustes correspondientes a las curvas de luz que se generaran con los datos, y por ultimo se utilizar\'a el modelo de \citet{2013Zharikov-model-catalysmic} para reproducir las curvas de luz en las bandas correspondientes.

\chapter{DESARROLLO}

\section*{CAP\'ITULO 1: CONCEPTOS PRELIMINARES}
Se presentaran conceptos esenciales para contextualizar los objetos que se estudiaran en este trabjao, esto es: Estrellas de neutrones, enanas blancas, p\'ulsares, RedBacks y Black Widows. Se har\'a especial énfasis en las características de los sitemas binarios que contiene p\'ulsares así como en los sistemas del tipo Black Widow.

Por ultimo se especificaran los conceptos relativos a la medición de la radiación en el rango óptico, as\'i como la obtención y reducción  de datos.

\section*{CAP\'ITULO 2: INSTRUMENTOS DE OBSERVACIÓN}
Consistir\'a en la descripción de El Gran Telescopio Canarias, su historia, instrumentos e importancia para este trabajo.

\section*{CAP\'ITULO 3: REDUCCIÓN Y ANÁLISIS DE DATOS}
Se analizar\'a la información obtenida en tres filtros Sloan $g', r'$ e $i'$, utilizando el software ``Image Reduction and Analysis Facility'' (IRAF) y modelos de espectros para Black Widows. 

\section*{CAP\'ITULO 4: RESULTADOS}
Se adjuntaran las gráficas de espectro de potencia para la fuente variable y gráficas de curvas de luz en los filtros $g', r'$ e $i'$.

\chapter{CONTENIDOS}

\noindent \textbf{LISTA DE FIGURAS}\\
\textbf{LISTA DE CONTENIDOS}\\
\textbf{LISTA DE SÍMBOLOS}\\
\textbf{OBJETIVOS}\\
\textbf{INTRODUCCIÓN}
\begin{enumerate}[leftmargin=0.5cm, label=\textbf{\arabic*}.]
    \item \textbf{CONCEPTOS PRELIMINARES}
    \begin{enumerate}[label=1.\arabic*.]
        \item OBJETOS COMPACTOS
        \begin{enumerate}[label=1.1.\arabic*.]
            \item Enanas Blancas
            \item Estrellas de Neutrones
            \item P\'ulsar
            \item P\'ulsar de Milisegundo
            \item P\'ulsares Reciclados
        \end{enumerate}
        \item RedBacks
        \item Black Widows P\'ulsares
        \item Sistema Binario Eclípsante
        \item OBSERVACIONES ÓPTICAS
        \begin{enumerate}[label=1.5.\arabic*.]
            \item Sistema de magnitudes
            \item Estrellas estándar
            \item Reducción básica de datos
        \end{enumerate}
    \end{enumerate}
    \item \textbf{INSTRUMENTOS DE OBSERVACIÓN}
    \begin{enumerate}[label=2.\arabic*.]
        \item GRAN TELESCOPIO CANARIAS
    \end{enumerate}
    
    \item \textbf{REDUCCIÓN Y ANÁLISIS DE DATOS}
    \begin{enumerate}[label=3.\arabic*.]
        \item METODOLOGÍA
    \end{enumerate}
    \item \textbf{RESULTADOS}
\end{enumerate}
\textbf{CONCLUSIONES}\\
\textbf{RECOMENDACIONES}\\
\textbf{BIBLIOGRAFÍA}

